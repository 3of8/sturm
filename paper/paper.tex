\documentclass[11pt,a4paper,oneside]{article}
\usepackage[english]{babel}

\usepackage[utf8x]{inputenc} 
\usepackage[T1]{fontenc}

\usepackage{geometry} 
\usepackage{color} 
\usepackage{graphicx} 
\usepackage{pifont} 
\usepackage[babel]{csquotes}
\usepackage{textcomp}
\usepackage{upgreek}
\usepackage{amsmath,amssymb,amsthm}
\usepackage{textcomp}
\usepackage{latexsym}
\usepackage{pgf}
\usepackage{nicefrac}
\usepackage{enumerate}
\usepackage{xspace}
\usepackage{stmaryrd}

\usepackage{tgpagella}

\DeclareFontFamily{OT1}{pzc}{}
\DeclareFontShape{OT1}{pzc}{m}{it}{<-> s * [1.10] pzcmi7t}{}
\DeclareMathAlphabet{\mathpzc}{OT1}{pzc}{m}{it}
\DeclareUnicodeCharacter{10230}{$\rightarrow$}

\newtheorem{definition}{Definition}
\newtheorem{lemma}[definition]{Lemma}
\newtheorem{theorem}[definition]{Theorem}
\newcommand{\flush}{\leavevmode\newline\vspace*{-1.5em}}

%\newcommand{\qed}[0]{\vspace{-3mm}\begin{flushright}\textit{q.e.d.}\end{flushright}\vspace{3mm}}
\newcommand{\bred}{\ensuremath{\longrightarrow_\beta}}
\newcommand{\acos}{\textit{arccos}}
\newcommand{\determ}[1]{\textit{det}(#1)}
\newcommand{\RR}{\mathbb{R}}
\newcommand{\BB}{\mathbb{B}}
\newcommand{\NN}{\mathbb{N}}
\newcommand{\QQ}{\mathbb{Q}}
\newcommand{\ZZ}{\mathbb{Z}}
\newcommand{\CC}{\mathbb{C}}
\newcommand{\II}{\mathbb{I}}
\newcommand{\kernel}[1]{\textit{ker}(#1)}
\renewcommand{\epsilon}{\varepsilon}
\renewcommand{\phi}{\varphi}
\renewcommand{\theta}{\vartheta}
\newcommand{\atan}{\mathrm{arctan}}
\newcommand{\rot}{\mathrm{rot}}
\newcommand{\vdiv}{\mathrm{div}}
\newcommand{\shouldbe}{\stackrel{!}{=}}
\newcommand{\sgn}{\mathrm{sgn}}
\renewcommand{\mod}{\ \ \textrm{mod}\ \ }

\newcommand{\sturm}{\texttt{sturm}}
\newcommand{\ilemma}{\textbf{lemma}}
\newcommand{\card}{\textrm{card}}
\newcommand{\real}{\textrm{real}}

\newcommand{\ie}{i.\,e.\xspace}
\newcommand{\eg}{e.\,g.\xspace}
\newcommand{\wrt}{w.\,r.\,t.\xspace}

\newcommand{\isabellehol}{\mbox{Isabelle}\slash HOL\xspace}

\geometry{a4paper,left=30mm,right=30mm, top=25mm, bottom=30mm} 

\title{A Formalisation of Sturm's theorem in \isabellehol}
\author{Manuel Eberl}

\begin{document}
\maketitle
\tableofcontents
\newpage
\parindent0mm

\section{Outline}

This paper will give a detailed proof that closely follows the structure of the formal proof in \isabellehol, but omits the proofs for basic lemmas about analysis and polynomials.\\

Section \ref{sec:notation} introduces some basic notation for well-known concepts that are not directly related to Sturm sequence. Section \ref{sec:defs} then defines a number of basic concepts that are used in Sturm's theorem, in particular, the concept of a \emph{Sturm sequence}. Section \ref{sec:aux_lemmas} contains some simple auxiliary lemmas that will be required in the main proof. Section \ref{sec:roots_sturm} contains the main portion of the proof, as it shows that Sturm sequences can be used to count the real roots of polynomials. Section \ref{sec:construction} then shows how Sturm sequences can be constructed in different cases and how these constructions can be applied. Finally, section \ref{sec:applications} explains how further preprocessing can be used to automatically decide complex, interesting properties on real polynomials.

\section{Notation}
\label{sec:notation}
In our the following proofs, we will use the following notation and terminology:
\begin{itemize}
\item $\RR[X]$ denotes the set of all polynomials with real coefficients.
\item $P'$ denotes the derivative of a polynomial $P\in\RR[X]$.
\item Some property holds in a neighbourhood of some $x_0\in\RR$ if there exists an $\epsilon > 0$ such that the property holds for all $x\in (x_0-\epsilon;x_0+\epsilon)\setminus\{x_0\}$. Note that it does not have to hold \emph{at} $x_0$ \emph{itself}.
\item $\mathrm{lc}(P)$ denotes the leading coefficient of a polynomial $P$.
\end{itemize}


\newpage
\section{Proof of Sturm's theorem}

\subsection{Definitions}
\label{sec:defs}

First, we define three notions that will be important in the proof of Sturm's theorem:

A \emph{quasi-Sturm sequence} is a list of polynomials $P_0,P_1,\ldots,P_{n-1} \in \RR[X]$ that fulfils the following properties:
\begin{itemize}
\item $n \geq 1$, i.e. the sequence is not empty
\item $P_{n-1}$ does not change its sign, i.e. $\sgn(P_{n-1}(x))=\sgn(P_{n-1}(y))$ for all $x,y\in\RR$
\item for any $i\in\{0,\ldots,n-1\}$ and any root $x_0$ of $P_{i+1}$, $P_{i}(x_0)P_{i+2}(x_0) < 0$ holds
\end{itemize}
It can easily be seen that any nonempty sequence obtained by dropping a number of elements from the beginning of a quasi-Sturm sequence is again a quasi-Sturm sequence.\\

A \emph{Sturm sequence} is a quasi-Sturm sequence that fulfils the following additional properties:
\begin{itemize}
\item $n \geq 2$, i.e. the sequence contains at least two polynomials
\item for any root $x_0$ of $P_0$, $P_0(x)P_1(x)$ is negative in some sufficiently small interval\\ \mbox{$(x_0-\epsilon;x_0)$} and positive in some sufficiently small interval $(x_0;x_0+\epsilon)$\footnote{
In \isabellehol, this is expressed as \enquote{eventually, the property $\sgn(P_0P_1(x))\ =\ \mathsf{if}\ x>x_0\ \mathsf{then}\ 1\ \mathsf{else}\ -1$ holds at $x_0$}.}
\item $P_0$ and $P_1$ have no common roots.
\end{itemize}

In Isabelle, these two concepts are captured in \emph{locales} of the same name.
\vskip5mm

Next, we define the notion of \emph{sign changes}. Let $P_0,\ldots,P_{n-1}\in\RR[X]$ be a sequence of polynomials and $x\in\RR$. By evaluating the $P_i$ at $x$, we obtain a sequence of real numbers $y_0,\ldots,y_{n-1}$. We now traverse this sequence from left to right and count how often the sign changes, \ie how often we see a positive number and the last nonzero number in the sequence was negative (or vice versa). This is called the number of \emph{sign changes} of the sequence $P_0,\ldots,P_{n-1}$ at the position $x$ and is denoted as $\sigma(P_0,\ldots,P_{n-1}; x)$.\\
In Isabelle, this is realised by evaluating the $P_i$ at $x$, deleting all zeros from the list, applying the \emph{remdups\_adj} operation on the list and taking the length of the remaining list minus one as the result. The \emph{remdups\_adj} function simply deletes equal adjacent elements in the list, \eg it turns $[1,2,2,1,1,1]$ into $[1,2,1]$.

A related notion are the sign changes \enquote{at infinity}. For this, we compute the sequences
$$\left(\sgn\left(\lim\limits_{x\to -\infty} P_i(x)\right)\right)_{0\leq i < n}\hskip6mm\textit{and}\hskip10mm
\left(\sgn\left(\lim\limits_{x\to \infty} P_i(x)\right)\right)_{0\leq i < n}$$
and count the sign changes in these sequences in the same way as before. The number of sign changes is called $\sigma(P_0,\ldots,P_{n-1};-\infty)$ and $\sigma(P_0,\ldots,P_{n-1};\infty)$, respectively. Note that the signs of these limits for a $P_i$ can be determined by taking the sign of the leading coefficient of $P_i$.\footnote{For $\infty$ in general and for $-\infty$ and even degree, it is simply the sign of the leading coefficient; for $-\infty$ and odd degree, it is the opposite of the sign of the leading coefficient.}

\subsection{Important auxiliary lemmas}
\label{sec:aux_lemmas}

Since Sturm's theorem is based on counting sign changes, a crucial part in proving it is to formalise the counting of sign changes in a way that is convenient for subsequent proofs. In this proof, this was done by proving that under certain conditions, the computation of the number of sign changes of a sequence can be \enquote{decomposed}.

\begin{lemma}\label{thm:limits}
Let $P\in\RR[X]\setminus\{0\}$. Then there exist $l,u\in\RR$ such that $l<u$, all roots of $P$ are in $(l;u)$ and
\begin{align*}
&\forall x\leq l.\ \sgn(P(x)) = \sgn\left(\lim\limits_{x\to -\infty} P(x)\right)\\
&\forall x\geq u.\ \sgn(P(x)) = \sgn\left(\lim\limits_{x\to \infty} P(x)\right)
\end{align*}
\end{lemma}
\begin{proof}
Obviously, the limit of $P(x)$ at $-\infty$ (resp. $\infty$) is either $\infty$ or $-\infty$. Therefore, there exists an $l$ (resp. a $u$) such that for all $x\leq l$ (resp. $x\geq u$), $P(x)$ increases above every bound in the case of the limit $\infty$ or falls below every bound in case of the limit $-\infty$. In particular, if we choose 0 as the bound, we see that $P(x)$ will be positive for all values beyond some point in the case of $\infty$ or negative in the case of $-\infty$, which implies $\sgn(P(x)) = \sgn(\lim_{x\to -\infty} P(x))$ (resp. $\sgn(P(x)) = \sgn(\lim_{x\to \infty} P(x))$) for all $x$ beyond that point. Therefore, bounds $l$ and $u$ that fulfil the last two requirements exist.\\
Also, since the $P(x)$ is positive or negative for all $x$ below $l$ (resp. above $u$), it has no roots outside the interval $(l;u)$. Lastly, should $l<u$ not hold already, we can simply increase $u$ until it does.
\end{proof}\vskip3mm

\begin{lemma}\label{thm:sign_changes_distrib}
Let $P_0, \ldots, P_{n-1}$ be a sequence of polynomials and $i\in\{0,\ldots,n-1\}$ and $x\in\RR$. Furthermore, assume $P_i(x) \neq 0$. Then $\sigma(P_0,\ldots,P_{n-1}; x) = \sigma(P_0,\ldots,P_i; x)+\sigma(P_i,\ldots,P_{n-1}; x)$
\end{lemma}
\begin{proof}
According to the informal definition of sign changes, this is obvious; for the Isabelle definition using list operation, it can be proven using simple properties of the \emph{remdups\_adj} and \emph{filter} functions.
\end{proof}\vskip3mm

\begin{lemma}\label{thm:sign_changes_triple}
Let $P,Q,R\in\RR[X]$ and $x\in\RR$ with $P(x)\neq 0$ and $\sgn(R(x)) = -\sgn(P(x))$. Then $\sigma(P,Q,R; x) = 1$
\end{lemma}
\begin{proof}
By case distinction
\end{proof}\vskip3mm

\begin{lemma}\label{thm:same_signs}
Let $P_0,\ldots,P_{n-1}\in\RR[X]$, $Q_0\ldots,Q_{n-1}\in\RR[X]$ be two sequences of polynomials and $x\in\RR$. Assume $\sgn(P_i(x_1))=\sgn(Q_i(x_2))$ for all $i$. Then $\sigma(P_0,\ldots,P_{n-1};x_1)=\sigma(Q_0,\ldots,Q_{n-1};x_2)$.
\end{lemma}
\begin{proof}
Trivial.
\end{proof}\vskip3mm

Later, we will also require some algebraic properties of polynomials, such as squarefree decomposition of polynomials:
\begin{lemma}\label{thm:squarefree}
Let $P \in \RR[X]\setminus \RR$, i.e. a nonconstant real polynomial. Let $D:=\gcd(P,P')$. Then the following holds:
\begin{enumerate}
\item $\gcd(P/D, P'/D) = 1$
\item $\forall x\in\RR.\ (P/D)(x)=0 \Longleftrightarrow P(x)=0$, \ie $P$ and $P/D$ have the same roots, disregarding multiplicity
\end{enumerate}
\end{lemma}\vskip3mm

%\begin{lemma}
%Let $P_0,\ldots,P_{n-1}\in\RR[X]$ be a quasi-Sturm sequence of length $\geq 3$. Then no two adjacent polynomials $P_i, P_{i+1}$ have a common root.
%\end{lemma}
%\begin{proof}
%Suppose $P_i$ and $P_{i+1}$ had a common root $x$. Since $n \geq 3$, at least one of $P_{i-1}$ or $P_{i+2}$ exists, and with the definition of a quasi-Sturm sequence and the fact that $P_{i-1}(x)=0$ and $P_i(x)=0$, we can conclude $0=P_{i-1}P_{i+1}<0$ resp. $0=P_iP_{i+2}<0$, which is obviously false.
%\end{proof}

\subsection{Relating roots and Sturm sequences}
\label{sec:roots_sturm}

We will now show how Sturm sequences can be used to determine the number roots of polynomials in a given interval. To this end, we will first explore how the number of sign changes of a Sturm sequence behaves in the neighbourhood of non-roots and roots of the polynomial.\\

\begin{lemma}\label{thm:nonzero}
Let $P_0,\ldots,P_{n-1}$ be a quasi-Sturm sequence and $x_0\in\RR$. Assume $P_0(x_0)\neq 0$. Then $\sigma(P_0,\ldots,P_{n-1}; x_0)$ is constant in the neighbourhood of $x_0$.
\end{lemma}
\begin{proof}
By induction over the length of the sequence
\begin{itemize}
\item if the sequence has length 1, the number of sign changes is, of course, $0$ at any position.
\item if the sequence has length 2, we know that $P_1$ does not change its sign anywhere by definition of a quasi-Sturm sequence. Furthermore, since $P_0(x_0)\neq 0$, the sign of $P_0$ does not change in the neighbourhood of $x_0$. By Lemma \ref{thm:same_signs}, the number of sign changes of the sequence $P_0,P_1$ also remains constant in the neighbourhood of $x_0$.
\item if the sequence has length $\geq 3$ and $P_1(x_0)\neq 0$, Lemma \ref{thm:sign_changes_distrib} implies $$\sigma(P_0,\ldots,P_{n-1}; x_0)=\sigma(P_0,P_1;x_0)+\sigma(P_1,\ldots,P_{n-1};x_0)$$ Since $P_0(x_0)\neq 0$ and $P_1(x_0)\neq 0$, the signs of $P_0$ and $P_1$ do not change in the neighbourhood of $x_0$ and therefore $\sigma(P_0, P_1;x_0)$ is constant in the neighbourhood of $x_0$. Also, the sequence $P_1,\ldots,P_{n-1}$ is again a quasi-Sturm sequence with $P_1(x_0)\neq 0$, \ie the induction hypothesis can be applied and we can conclude that $\sigma(P_1,\ldots,P_{n-1};x_0)$ is also constant in the neighbourhood of $x_0$.
\item if the sequence has length $\geq 3$ and $P_1(x_0)=0$, we have $P_0(x_0)P_2(x_0)<0$ by the definition of a quasi-Sturm sequence and thus $P_0(x_0)\neq 0$, $P_2(x_0)\neq 0$, and $\sgn(P_2(x_0))=-\sgn(P_0(x_0))$. With Lemma \ref{thm:sign_changes_distrib}, we have: $$\sigma(P_0,\ldots,P_{n-1}; x_0) = \sigma(P_0,P_1,P_2;x_0) + \sigma(P_2,\ldots,P_{n-1}; x_0)$$ Furthermore, since $P_0(x_0)\neq 0$ and $P_2(x_0)\neq 0$, $\sgn(P_2(x))=-\sgn(P_0(x))$ holds in the entire neighbourhood of $x_0$ as the signs of $P_0$ and $P_2$ do not change in the neighbourhood of $x_0$, and with Lemma \ref{thm:sign_changes_triple}, we then have $\sigma(P_0,P_1,P_2;x)=1$ (\ie constant) in the entire neighbourhood of $x_0$. As for the second summand, we know that $P_2,\ldots,P_{n-1}$ is again a quasi-Sturm sequence and $P_2(x_0)\neq 0$, so we can apply the induction hypothesis and obtain that $\sigma(P_2,\ldots,P_{n-1};x_0)$, too, is constant in the neighbourhood of $x_0$.
\end{itemize}
\end{proof}

\newpage
\label{thm:zero}
\begin{lemma}
Let $P_0,\ldots,P_{n-1}$ be a Sturm sequence and $x_0\in\RR$. Assume $P_0(x_0)=0$. Then for $x$ in the neighbourhood of $x_0$, we have $\sigma(P_0,\ldots,P_{n-1};x)=\sigma(P_0,\ldots,P_{n-1};x_0)+1$ if $x<x_0$ and $\sigma(P_0,\ldots,P_{n-1};x)=\sigma(P_0,\ldots,P_{n-1};x_0)$ if $x\geq x_0$.
\end{lemma}
\begin{proof}
From the definition of a Sturm sequence, we know that $P_0(x_0)=0$ implies $P_1(x_0)\neq 0$. Therefore, $P_1(x_0)$ has the same nonzero sign in a neighbourhood of $x_0$. Moreover, in a neighbourhood of $x_0$, we have $P_0(x)P_1(x)<0$ if $x<x_0$ and $P_0(x)P_1(x)>0$ if $x>x_0$. With Lemma \ref{thm:sign_changes_distrib}, we have:
$$\hskip1mm\sigma(P_0,\ldots,P_{n-1};x)=\sigma(P_0,P_1;x)+\sigma(P_1,\ldots,P_{n-1};x) \hskip10mm\mathrm{in\ a\ NH\ of\ }x_0$$
Since $P_1(x_0)\neq 0$ in a neighbourhood of $x_0$, we can apply Lemma \ref{thm:nonzero} and obtain:
$$\hskip12mm\sigma(P_1,\ldots,P_{n-1};x) = \sigma(P_1,\ldots,P_{n-1};x_0) \hskip21mm\mathrm{in\ a\ NH\ of\ }x_0$$
Of course, since $P_0(x_0)=0$, we also have:
$$\sigma(P_1,\ldots,P_{n-1};x_0) = \sigma(P_0,\ldots,P_{n-1};x_0)\hskip31mm$$
In summary, we have shown so far that:
$$\sigma(P_0,\ldots,P_{n-1};x)=\sigma(P_0,P_1;x) + \sigma(P_0,\ldots,P_{n-1};x_0) \hskip10mm\mathrm{in\ a\ NH\ of\ }x_0$$
Therefore, what remains to be shown is that $\sigma(P_0,P_1;x)=1$ for $x<x_0$ and $\sigma(P_0,P_1;x)=0$ for $x>x_0$ for all $x$ in a neighbourhood of $x_0$. This is a simple of $P_0(x)P_1(x)<0$ for $x<0$ and $P_0(x)P_1(x)>0$ for $x>0$ and the fact that $P_1$ has the same sign in the entire neighbourhood of $x_0$.
\end{proof}\vskip5mm

To express it in a less formal way: when passing through the real numbers in increasing order and tracking the number of sign changes of a Sturm seuqnece of a real polynomial $P$, that number increases by $1$ every time one passes a root of $P$ and remains constant otherwise. Intuitively, it should now be clear that Sturm sequences can be used to count the number of roots of a polynomial in a given interval. We will now prove this formally.


\label{thm:count_roots_between}
\begin{lemma}
Let $P_0,\ldots,P_{n-1}$ be a Sturm sequence with $P_0\neq 0$ and $x_0\in\RR$. Fix $a,b\in\RR$ with $a \leq b$. Then the following holds:
$$\sigma(P_0,\ldots,P_{n-1}; a)-\sigma(P_0,\ldots,P_{n-1}; b) = |\{x\in (a;b]\ |\ P_0(x)=0\}|$$
\end{lemma}
\begin{proof}
By induction over $|\{x\in (a;b].\ P_0(x)=0\}|$ for arbitrary $a, b$.
\begin{itemize}
\item if $|\{x\in (a;b].\ P_0(x)=0\}|=0$, we have $P_0(x)\neq 0$ for all $x\in (a;b]$. We now distinguish two cases:
\begin{itemize}
\item if $P_0(a)\neq 0$, we have $P_0(x)\neq 0$ for all $x\in[a;b]$. Lemma \ref{thm:nonzero} then implies that $\sigma(P_0,\ldots,P_{n-1})$ is constant in the neighbourhood of all $x\in[a;b]$ and must therefore be constant in the entire interval $[a;b]$.
\item if $P_0(a)=0$, we know from Lemma \ref{thm:zero} that $$\sigma(P_0,\ldots,P_{n-1}; x)=\sigma(P_0,\ldots,P_{n-1}; a)$$ for all $x\in(a;a+\epsilon)$ for some $\epsilon > 0$, \ie $\sigma(P_0,\ldots,P_{n-1})$ is constant in the right neighbourhood of $a$. Furthermore, using $P_0(x)\neq 0$ for all $x\in(a;b]$ and Lemma \ref{thm:nonzero}, that $\sigma(P_0,\ldots,P_{n-1})$ is constant in the neighbourhood of every $x\in(a;b]$. Therefore, $\sigma(P_0,\ldots,P_{n-1})$ is constant on the entire interval $[a;b]$.
\end{itemize}
Since $\sigma(P_0,\ldots,P_{n-1})$ is constant on the entire interval $[a;b]$, we have $$\sigma(P_0,\ldots,P_{n-1}; a)-\sigma(P_0,\ldots,P_{n-1}; b) = 0$$ which is exactly the number of roots of $P_0$ in $(a;b]$.

\item if $|\{x\in (a;b].\ P_0(x)=0\}|=n>0$, we take the smallest root of $P_0$ in the interval $(a;b]$ and call it $x_2$. With Lemma \ref{thm:zero}, we know that $\sigma(P_0,\ldots,P_{n-1};x) = \sigma(P_0,\ldots,P_{n-1};x_2)+1$ for all $x$ in a sufficiently small left neighbourhood of $x_2$. Let $x_1$ then be some value in $(a;x_2)$ that is in this neighbourhood. We know that $P_0$ has no roots in the interval $(a;x_2]$ and thus, by induction hypothesis, $\sigma(P_0,\ldots,P_{n-1};a)=\sigma(P_0,\ldots,P_{n-1};x_1)=\sigma(P_0,\ldots,P_{n-1};x_2)+1$. Furthermore, the number of roots of $P_0$ in the interval $(x_2;b]$ is exactly $n-1$, so by induction hypothesis, $\sigma(P_0,\ldots,P_{n-1};x_2)-\sigma(P_0,\ldots,P_{n-1};b) = n-1$. Adding these two equations yields $\sigma(P_0,\ldots,P_{n-1};a)-\sigma(P_0,\ldots,P_{n-1};b) = n$.
\end{itemize}
\end{proof}



\begin{lemma}\label{thm:count_roots}
Let $P_0,\ldots,P_{n-1}$ be a Sturm sequence with $P_0\neq 0$. Then the following holds: $$\sigma(P_0,\ldots,P_{n-1}; -\infty)-\sigma(P_0,\ldots,P_{n-1}; \infty) = |\{x\ |\ P_0(x)=0\}|$$
\end{lemma}
\begin{proof}
Using Lemma \ref{thm:limits} for every $P_i$, we obtain $l,u\in\RR$ with $l<u$ such that $P_0$ has no roots outside the interval $(l;u)$ and for all $P_i$, $\sgn(P_i(l))=\sgn(\lim_{x\to -\infty} P_i(x))$ and $\sgn(P_i(u))=\sgn(\lim_{x\to \infty} P_i(x))$. It is then easy to see that the number of sign changes of the sequence $P_0,\ldots,P_{n-1}$ can be determined by considering the signs of the limits of the $P_i$ at $\pm\infty$ instead of $P_i(u)$ and $P_i(l)$. Since all roots of $P_0$ lie in the interval $(l;u)$, we have thus shown that the total number of roots of $P_0$ can be determined in the way described above.
\end{proof}

Note: similar statements for the usage of $\sigma$ at $\pm\infty$ to compute the number of roots $>a$, $<a$, $\geq a$, or $\leq a$ for some fixed $a\in\RR$ can be proven analogously.

\newpage
\section{Constructing Sturm sequences}
\label{sec:construction}
\subsection{The canonical Sturm sequence}

The canonical Sturm sequence $P_0,\ldots,P_{n-1}$ of a polynomial $P\in\RR[X]$ is constructed as follows:
\begin{align*}
&&\mathrm{where}\quad &P_i = \begin{cases}
\hskip3mm P & \mathrm{for}\ i = 0\\
\hskip3mm P' & \mathrm{for}\ i = 1\\
-P_{i-2}\ \mathrm{mod}\ P_{i-1} &\mathrm{otherwise}
\end{cases}&
\end{align*}

$n$ is chosen such that $n\geq 2$ and $P_{n-1}$ is constant, since this will simplify formalisation. Choosing a higher value for $n$ only results in a number of zero polynomials at the end of the sequence, which do not change the result in any way. Note that this construction always terminates as the degree of the polynomials involved strictly decreases with every step, so that one reaches a constant polynomial in finitely many steps.\\

\begin{lemma}\label{thm:sturm_gcd}
For all $i\in\{0,n-1\}$, we have $\gcd(P_i, P_{i+1}) = \gcd(P,P')$
\end{lemma}
\begin{proof}
By induction on $i$ using the fact that $$\gcd(R,S)=\gcd(S,R\mod S)=\gcd(S,-R\mod S)$$\vskip-3mm
\end{proof}\vskip3mm

\begin{lemma}\label{thm:sturm_seq_canonical}
Let $P\in\RR[X]$ with no multiple roots. Then the canonical Sturm sequence construction of $P$ yields an actual Sturm sequence.
\end{lemma}
\begin{proof}
We simply prove the five conditions a Sturm sequence has to satisfy:
\begin{itemize}
\item the sequence has at least length 2: obvious by definition
\item $P_{n-1}$ does not change its sign: obvious, since $P_{n-1}$ is constant
\item for any $i\in\{0,\ldots,n-1\}$ and any root $x_0$ of $P_{i+1}$, $P_{i}(x_0)P_{i+2}(x_0) < 0$ holds:\\
By construction, we have $P_{i+2} = -P_{i}\ \textit{mod}\ P_{i+1}$. With $P_{i+1}(x_0)=0$, this implies that $P_{i+2}(x) = -P_i(x)$, and because of Lemma \ref{thm:sturm_gcd} and the fact that $P$ has no multiple roots, we also have $P_i(x)\neq 0$ and thus $P_i(x)P_{i+2}(x)<0$.
\item for any root $x_0$ of $P_0$, $P_0(x)P_1(x)$ is negative in some sufficiently small interval\\ \mbox{$(x_0-\epsilon;x_0)$} and positive in some sufficiently small interval $(x_0;x_0+\epsilon)$:\\
Since $P_0=P$ and $P_1=P'$ are polynomials, there exists some neighbourhood of $x_0$ that does not contain any roots of either $P_0$ or $P_1$. For any $x<x_0$ in that neighbourhood, we can then apply the mean value theorem to obtain some $\xi\in[x;x_0]$ with $P(x)-P(x_0) = P'(\xi)(x-x_0)$, and since $P(x_0)=0$ and $x-x_0<0$, $P(x)=P'(\xi)$. Furthermore, $P(x)\neq 0$ and $P'(\xi)$ has the same sign as $P'(x)$, since we chose a neighbourhood without roots; therefore we have $P(x)P'(x)<0$. For any $x>x_0$, we can use the same argument to find that $P(x)P'(x)>0$.
\item $P_0$ and $P_1$ have no common roots:\\
$P_0=P$ and $P_1=P'$, so if $P_0$ and $P_1$ had a common root, it would be a multiple root of $P$, which contradicts our assumption.
\end{itemize}
\end{proof}
%First of all, it can easily be seen that some $x_0\in\RR$ is a multiple root of $P$ iff it is a root of both $P$ and $P'$ and thus a root of $\gcd(P,P')$:
%\begin{itemize}
%\item if $x_0$ is a root of $P$ with order $k\geq 2$, we can write $P$ as $(x-x_0)^kQ$ for some $Q\in\RR[X]$, and thus $P'=((x-x_0)^kQ)' = (x-x_0)^{k-1}Q + (x-x_0)^kQ' = (x-x_0)^{k-1}(Q + (x-x_0)Q')$, \ie $x_0$ is also a root of order $k-1\geq 1$ of $P'$. Consequently, $x-x_0$ divides both $P$ and $P'$ and thus also $\gcd(P,P')$, so that $x_0$ is a root of $\gcd(P,P')$.
%\item if $x_0$ is a root of $\gcd(P,P')$, $x-x_0$ divides $\gcd(P,P')$ and thus divides $P$ and $P'$. Therefore, $x_0$ is a root of both $P$ and $P'$. We can then write $P$ as $(x-x_0)Q$ for some $Q\in\RR[X]$ and then have $P'=((x-x_0)Q)'= Q + (x-x_0)Q'$, and since $x-x_0$ divides $P'=Q+(x-x_0)Q'$, it must also divide $Q$, which implies that $x_0$ is a root of $P$ of at least order 2.
%\end{itemize}


\subsection{The case of multiple roots}

Of course, one way to handle a polynomial with multiple roots is to use lemma \ref{thm:squarefree} and \enquote{divide away} the \enquote{excess roots} by dividing the polynomial $P$ by $\gcd(P,P')$. However, surprisingly, it turns out that as long as the interval bounds $a$ and $b$ are not multiple roots of $P$ themselves, i.e. at least one of $P(a)\neq 0$ and $P'(a)\neq 0$ and at least one of $P(b)\neq 0$ and $P'(b)\neq 0$ holds, one can simply use the canonical Sturm sequence without any modification. To show this, we will first prove the following auxiliary result:

\begin{lemma}\label{thm:sturm_multiple_aux}
Let $P\in\RR[X]\setminus\RR$, i.e. a nonconstant real polynomial. Let $P_0,\ldots,P_{n-1}$ be the result of the canonical Sturm sequence construction of $P$ and define $Q_i := P_i / \gcd(P,P')$. Then $Q_i$ is a Sturm sequence and $Q_0$ has the same roots, disregarding multiplicity, as $P$.
\end{lemma}
\begin{proof}
First, we note that $\gcd(P,P')\neq 0$, since $P'\neq 0$ by assumption. Furthermore, lemma \ref{thm:sturm_gcd} implies that $\gcd(P,P')\ |\ P_i$ for all $i$, thus $Q_i\in\RR[X]$, i.e. the $Q_i$ really are polynomials. This means that our definition of $Q_0,\ldots,Q_{n-1}$ is well-defined.\\
Now we show that the five properties for a Sturm sequence are satisfied:
\begin{itemize}
\item the sequence has at least length 2: obvious by construction of $Q_0, \ldots, Q_{n+1}$
\item $Q_{n-1}$ does not change its sign: obvious, since $P_{n-1}$ is constant, so $Q_{n-1}$ is, too.
\item $Q_0$ and $Q_1$ have no common roots:\\
By construction, we have $Q_0 = P / \gcd(P,P')$ and $Q_1 = (P / \gcd(P, P'))'$. This then implies $\gcd(Q_0, Q_1) = 1$. However, the proof for this algebraic fact is nontrivial; it was, of course, proven in the \emph{Isabelle} version of this proof, but it is too complex and not important enough to be discussed in detail here.
\item for any $i\in\{0,\ldots,n-1\}$ and any root $x_0$ of $Q_{i+1}$, $Q_{i}(x_0)Q_{i+2}(x_0) < 0$ holds:\\
By construction of the canonical Sturm sequence, we have:
$$P_{i+2} = -P_{i}\ \mathrm{mod}\ P_{i+1}$$
Therefore, we have:
$$(P_{i}\ \mathrm{div}\ P_{i+1}) \cdot P_{i+1} - P_{i+2} = P_{i}$$
Let $D:=\gcd(P,P')$. Since $D$ divides all the $P_j$; we have $P_{i}=D\cdot Q_{i}$, $P_{i+1}=D\cdot Q_{i+1}$, and $P_{i+2}=D\cdot Q_{i+2}$ and thus:
$$(D\cdot Q_{i}\ \mathrm{div}\ D\cdot Q_{i+1}) \cdot D\cdot Q_{i+1} - D\cdot Q_{i+2} = D\cdot Q_{i}$$
Cancelling $D$ yields:
$$(Q_{i}\ \mathrm{div}\ Q_{i+1}) \cdot Q_{i+1} - Q_{i+2} = Q_{i}$$
Since $x_0$ is a root of $Q_{i+1}(x_0)$, we then have:
$$-Q_{i+2}(x_0) = Q_{i}(x_0)$$
Since $\gcd(Q_{i}, Q_{i+1}) = 1$ and $x_0$ is a root of $Q_{i+1}(x_0)$, $x_0$ cannot be a root of $Q_{i}$ and thus we have:
$$Q_{i}(x_0)Q_{i+2}(x_0) = -Q_{i}(x_0)^2 < 0$$

\newpage
\item for any root $x_0$ of $Q_0$, $Q_0(x)Q_1(x)$ is negative in some sufficiently small interval\\ \mbox{$(x_0-\epsilon;x_0)$} and positive in some sufficiently small interval $(x_0;x_0+\epsilon)$:\\
Let, again, $D:=\gcd(P, P')$. Obtain some an $\epsilon>0$ such that the following two properties hold:
\begin{enumerate}
\item $D$ does not have a root in $(x_0-\epsilon; x_0+\epsilon)\setminus\{x_0\}$
\item $P_0(x)P_1(x)<0$ for all $x\in (x_0-\epsilon; x_0)$ and $P_0(x)P_1(x)>0$ for all $x\in (x_0;x_0+\epsilon)$ % TODO
\end{enumerate}
Then we have, for any $x$ in $(x_0-\epsilon; x_0+\epsilon)\setminus\{x_0\}$:
$$P_0(x)P_1(x) = Q_0(x)Q_1(x)\cdot \underbrace{D(x)^2}_{>0}$$
This then obviously implies the property we want to show.
\end{itemize}
\vskip3mm
It remains to show that $Q_0$ has the same roots, disregarding multiplicity, as $P = P_0$. This is precisely the statement of lemma \ref{thm:squarefree}.
\end{proof}\vskip10mm

Of course, there is no reason to use this construction in practice, since if one has computed $\gcd(P,P')$ already, it is easier to compute the canonial Sturm sequence of $P/\gcd(P,P')$ directly than to compute the canonical Sturm sequence of $P$ and then divide every polynomial in it by $\gcd(P,P')$. However, it becomes useful when combined with the following insight:
\begin{lemma}\label{thm:sturm_multiple_aux2}
Let $P\in\RR[X]\setminus\RR$, i.e. a nonconstant real polynomial. Let $P_0, \ldots, P_{n-1}$ and $Q_0,\ldots, Q_{n-1}$ be as in the previous lemma. Then the following holds:
\begin{enumerate}
\item $\forall x\in\RR.\ P(x)\neq 0\vee P'(x)\neq 0 \Longrightarrow \sigma(Q_0, \ldots, Q_{n-1}; x) = \sigma(P_0, \ldots, P_{n-1}; x)$
\item $P\neq 0 \Longrightarrow \sigma(Q_0,\ldots,Q_{n-1}; \pm\infty) = \sigma(P_9,\ldots,P_{n-1};\pm\infty)$
\end{enumerate}
\end{lemma}
\begin{proof}\ 
\begin{enumerate}
\item Let $D:=\gcd(P,P')$. Consider an arbitrary $x\in\RR$ for which $P(x)\neq 0$ or $P'(x)\neq 0$, \ie $D(x)\neq 0$. Then we have for all $i$:
$$\mathrm{sgn}(Q_i(x)) = \mathrm{sgn}(P_i(x) / D(x)) = \mathrm{sgn}(P_i(x)) \cdot \mathrm{sgn}(D(x))$$
It is now easy to see that in both of the two cases $D(x)>0$ and $D(x)<0$, the number of sign changes in the two sequences is the same, since the signs in $Q_0,\ldots, Q_{n-1}$ are all either the same as in $P_0,\ldots, P_{n-1}$ or all flipped \wrt the signs in $P_0,\ldots,P_{n-1}$.
\item We now consider the case of $\pm\infty$. Since $P_i = Q_i\cdot D$, we obviously have $\mathrm{lc}(P_i) = \mathrm{lc}(D)\cdot \mathrm{lc}(Q_i)$. When considering the way in which $\sigma$ at $\pm\infty$ can be computed using the leading coefficients, it is then again obvious that the two sign sequences are either the same or one is \enquote{flipped} \wrt the other; in either way, we have $\sigma(Q_0,\ldots,Q_{n-1};\pm\infty) = \sigma(P_0,\ldots,P_{n-1};\pm\infty)$.
\end{enumerate}
\end{proof}

In conclusion: if we want to count the roots of some $P$ between bounds $a,b\in\RR\cup\{-\infty,\infty\}$ with $a\leq b$, we use $\sigma$, $\sigma_\infty$ and $\sigma_{-\infty}$ with a Sturm sequence of $P$, and such a Sturm sequence can be obtained by applying the canonical Sturm sequence construction to $P$ if neither $a$ nor $b$ are roots of both $P$ and $P'$, or by applying it to $P/\gcd(P,P')$ if they are.


\newpage
\section{Applications}
\label{sec:applications}

The statements that were proven in lemma \ref{thm:count_roots_between} and lemma \ref{thm:count_roots} were that Sturm sequences can be used to count the roots of a polynomial $P$ in an interval of the form $(a;b)$ for $a,b\in\RR$, $a < b$ or in the interval $(-\infty;\infty)$. However, by adding case distinctions on whether $a$ and $b$ are roots of $P$, this can easily be generalised to arbitrary open, half-closed, closed, bounded, and unbounded real intervals.

Furthermore, statements such as $\forall x \in I.\ P(x) \neq 0$ can be decided by counting the number of roots with the above method and checking whether it is zero. This, in turn, can be used to decide statements of the form $\forall x\in I.\ P(x) > 0$ by checking that $\forall x\in I.\ P(x) > 0$ holds and $\lim_{x\to\infty} P(x) = \infty$.

By observing that for any two polynomials $P$ and $Q$, we have 
\begin{align*}
P(x) = 0 \wedge Q(x) = 0\hskip5mm&\Longleftrightarrow\hskip5mm \gcd(P,Q) = 0\\
P(x) = 0 \vee Q(x) = 0\hskip5mm&\Longleftrightarrow\hskip5mm (P\cdot Q)(x) = 0
\end{align*}
we can further generalise our method to arbitrary statements of the form
$$(\forall x \in I.\ P(x))\hskip5mm\mathrm{and}\hskip5mm |\{x\ |\ Q(x)\}| = n$$
where $I$ is an open, half-closed, or closed, bounded or unbounded real interval and $P$ is a property consisting of the operators $\wedge$, $\vee$, $+$, $-$, $\cdot$, $\hat{\ }$, $\neq$ and $Q$ is a property consisting of the operators $\wedge$, $\vee$, $+$, $-$, $\cdot$, $\hat{\ }$, $=$.\\

Independently from this, the observation that $\forall x\in I.\ P(x)>0$ holds iff $P$ has no roots in $I$ and $P(x)>0$ for some arbitrary $x\in I$ also allows us to decide statements of the form
$$(\forall x\in I.\ P(x) > 0)\hskip5mm\mathrm{and}\hskip5mm(\forall x\in I.\ P(x) < 0)$$\vskip5mm

Concrete examples of statements that can be proved in \isabellehol by the \emph{sturm} method that implements the decision procedure we just explained are:
\footnote{For readers that are not familiar with Isabelle, \enquote{card} denotes the cardinality of a set, and \enquote{poly} constructs a polynomial from a list of coefficients, the first list element being the constant coefficient, the second one being the linear coefficient and so on. Moreover, in Isabelle, free variables in statements are implicitly universally quantified by convention.}
\begin{align*}
&\ilemma\ "\card\ \{x::\real.\ (x - 1)^2 * (x + 1) = 0\} = 2"\ \textrm{\textbf{by}\ sturm}\\
&\ilemma\ "\mathrm{card}\ \{x::\mathrm{real}.\ -0.010831 < x\ \wedge\ x < 0.010831\ \wedge\\
&\hskip20mm \mathrm{poly}\ [:0, -17/2097152, -49/16777216, 1/6, 1/24, 1/120:]\ \ x\ =\ 0\} = 3"\ \textrm{\textbf{by}\ sturm}\\
&\ilemma\ "\card\ \{x::\real.\ x^3 + x = 2*x^2\ \wedge\ x^3-6*x^2+11*x=6\} = 1"\ \textrm{\textbf{by}\ sturm}\\
&\ilemma\ "\card\ \{x::\real.\ x^3 + x = 2*x^2\ \vee\ x^3-6*x^2+11*x=6\} = 4"\ \textrm{\textbf{by}\ sturm}\\
&\ilemma\ "(x::\real)^2+1 > 0"\ \textrm{\textbf{by}\ sturm}\\
&\ilemma\ "(x::\real) > 0\ \Longrightarrow\ x^2+1 > 0"\ \textrm{\textbf{by}\ sturm}\\
&\ilemma\ "\llbracket (x::\real) > 0; x \leq 2/3\rrbracket\ \Longrightarrow\ x*x \neq\ x"\ \textrm{\textbf{by}\ sturm}\\
&\ilemma\ "(x::\real) > 1\ \Longrightarrow\ x*x > x"\ \textrm{\textbf{by}\ sturm}\\
\end{align*}

\end{document}

