\documentclass[11pt,a4paper,oneside]{article}
\usepackage[english]{babel}

\usepackage[utf8x]{inputenc} 
\usepackage[T1]{fontenc}

\usepackage{geometry} 
\usepackage{color} 
\usepackage{graphicx} 
\usepackage{pifont} 
\usepackage[babel]{csquotes}
\usepackage{textcomp}
\usepackage{upgreek}
\usepackage{amsmath}
\usepackage{textcomp}
\usepackage{amssymb}
\usepackage{latexsym}
\usepackage{pgf}
\usepackage{nicefrac}
\usepackage{enumerate}
\usepackage{xspace}

\usepackage{tgpagella}

\DeclareFontFamily{OT1}{pzc}{}
\DeclareFontShape{OT1}{pzc}{m}{it}{<-> s * [1.10] pzcmi7t}{}
\DeclareMathAlphabet{\mathpzc}{OT1}{pzc}{m}{it}
\DeclareUnicodeCharacter{10230}{$\rightarrow$}

\newcommand{\wuppdi}[0]{\hfill\ensuremath{\square}}
\newcommand{\qed}[0]{\vspace{-3mm}\begin{flushright}\textit{q.e.d.}\end{flushright}\vspace{3mm}}
\newcommand{\bred}{\ensuremath{\longrightarrow_\beta}}
\newcommand{\acos}{\textrm{arccos}}
\newcommand{\determ}[1]{\textrm{det}(#1)}
\newcommand{\RR}{\mathbb{R}}
\newcommand{\BB}{\mathbb{B}}
\newcommand{\NN}{\mathbb{N}}
\newcommand{\QQ}{\mathbb{Q}}
\newcommand{\ZZ}{\mathbb{Z}}
\newcommand{\CC}{\mathbb{C}}
\newcommand{\II}{\mathbb{I}}
\newcommand{\kernel}[1]{\textrm{ker}(#1)}
\renewcommand{\epsilon}{\varepsilon}
\renewcommand{\phi}{\varphi}
\renewcommand{\theta}{\vartheta}
\newcommand{\atan}{\mathrm{arctan}}
\newcommand{\rot}{\mathrm{rot}}
\newcommand{\vdiv}{\mathrm{div}}
\newcommand{\shouldbe}{\stackrel{!}{=}}

\newcommand{\isabellehol}{\mbox{Isabelle}\slash HOL\xspace}

\geometry{a4paper,left=30mm,right=30mm, top=25mm, bottom=30mm} 

\title{A Formalisation of Sturm's theorem in \isabellehol}
\author{Manuel Eberl}

\begin{document}
\maketitle
\vskip-3mm

\section{Background}

\emph{\isabellehol} is an interactive theorem proving environment based on 
\emph{Higher Order Logic}. The \isabellehol library already contains a 
theory of univariate polynomials, including, in particular, differentiability. 
Furthermore, all basic analysis, such as limits and the intermediate and mean value 
theorem, are also available. Thus, most of the basic ingredients to 
formalise Sturm's theorem are available.

To explain the statement of Sturm's theorem in its most special case, 
we will first introduce the notion of a Sturm sequence of a polynomial.\\
Let $p\in\RR[X]$ that has no multiple real roots. The \emph{Sturm sequence}, 
or \emph{Sturm chain}, of $p$ is then defined as the sequence 
$p_0,p_1,\ldots p_n$ where:
\begin{align*}
&&\mathrm{where}\quad &p_i = \begin{cases}
\hskip3mm p & \mathrm{for}\ i = 0\\
\hskip3mm p' & \mathrm{for}\ i = 1\\
-p_{i-2}\ \mathrm{mod}\ p_{i-1} &\mathrm{otherwise}
\end{cases}&\\[2mm]
&&\mathrm{and}\quad &n=\mathrm{min} \{i\in\NN\ |\ \mathrm{degree}(p_i) \leq 0\}&
\end{align*}

Further, let $\sigma(x)$ denote the number of sign changes in the sequence 
$p_0(x), p_1(x), \ldots, p_n(x)$ for any $x \in \RR$. 
By \enquote{sign change}, we mean that a sign at one position is $-1$ and at the 
next position, skipping all zeroes, the sign is $1$, or vice versa.

Sturm's theorem now states that for $a,b\in\RR$ with $a\leq b$, the number of 
real roots of $p$ in the interval $(a;b]$ is $\sigma(a)-\sigma(b)$.\footnote{
Since $p$ was assumed to not have multiple roots, multiplicity is not relevant 
for counting here.}

Based on this, a number of generalisations can be used to count roots of 
polynomials with multiple roots as well, and furthermore, the total number of 
real roots can be determined by using $\lim_{x\to\pm\infty} \sigma(x)$, which 
can be determined by looking at the leading coefficients of the polynomials in 
the chain.

\section{Goal}

The goal is to formalise Sturm's theorem and its aforementioned generalisations 
in \emph{\isabellehol}. The result should be an executable function for 
determining the number of roots of a polynomial. Such a function can be 
exported to a programming language such as ML or Haskell using Isabelle's 
code generator, or it can be used to implement a decision procedure fo the 
number of roots of a polynomial directly in Isabelle.\footnote{This can only work for rational coefficients and interval bounds, of course.}

One example for which such a decision procedure would be useful is the 
formalisation of algorithms for numerical approximation; for instance, there 
exists a formalisation of an algorithm for approximating the $\exp$ function 
that requires the fact that the polynomial 
$$\frac{1}{120}x^5+\frac{1}{24}x^4+\frac{1}{6}x^3-\frac{49}{16777216}x^2-\frac{17}{2097152}x$$
has exactly three roots in the interval $(-0.010831,0.010831)$. In Isabelle, this 
is written as:

\vspace*{-5mm}
\begin{align*}
&\textbf{lemma}\ "\mathrm{card}\ \{x::\mathrm{real}. -0.010831 < x\ \wedge\ x < 0.010831\ \wedge\\
&\hskip15mm 1/120*x^4 + 1/24*x^4 + 1/6*x^3 - 49/16777216*x^2 - 17/2097152*x\ =\ 0\} = 3"
\end{align*}

This fact is readily checkable with any computer algebra system, but was, so far, 
unfeasible to prove in Isabelle. We aim to be able to prove statements like these 
with our proof method derived from our formalisation of Sturm chains.


\section{Work outline}
We estimate that the work to be done is the following:
\begin{itemize}
\item Study the mathematical literature on Sturm chains
\item Define, with enough generality to allow for later generalisations and optimisations, what 
      properties a general Sturm chain must fulfil
\item Prove that a chain fulfilling these requirements can indeed be used to count roots
\item Prove that the various canonical and non-canonical sturm chains fulfil these requirements
\item Prove that one can determine the signs of polynomials for sufficiently large arguments by 
      looking at the leading coefficients (this may require extension of the Polynomial library,
      since no lemmas about limits of polynomials exist so far)
\item Use these definitions and proofs to implement a decision procedure as described above and 
      integrate it with Isabelle
\end{itemize}


\end{document}

